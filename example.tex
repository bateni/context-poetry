\usemodule[p-poetry]
\setupwhitespace[big]
\setupdirections[bidi=global,method=unicode]
\starttext

Here is \quotation{Amazing Grace} written by John Newton:

%% Common definition for all the English poetry settings below.
%\setuppoetry[before={%
%    \blackrule[width=\textwidth,height=.2ex]\page[no]},
%  after={\page[no]\blackrule[width=\textwidth,height=.2ex]}]
%%

\startpoetry[width=local,color=darkblue]
  \poetryline{Amazing Grace! how sweet the sound}{That saved a wretch like me;}
  \poetryline{I once was lost, but now am found;}{Was blind, but now I see.}
\stoppoetry

Here, the four \emph{hemistichs} all get the same width.
So extra spaces get added to get the alignment right.

In the above, we used \type{width=local}.  Since the default
\emph{desired} inter-hemistich distance is \type{4em}, the two hemistichs did not fit
on the line and automatically switched to the
\type{alternative=staggered} mode.

Let's reduce that distance and see that in \type{alternative=auto} or
\type{alternative=line} modes, we can also fit things in the line.

\startpoetry[width=local,distance=2\emwidth,color=darkblue]
  \poetryline{Amazing Grace! how sweet the sound}{That saved a wretch like me;}
  \poetryline{I once was lost, but now am found;}{Was blind, but now I see.}
\stoppoetry

This capability is the original reason I developed this module to
extend what was available in the \type{m-hemistich} module already.
We have three alternatives for typesetting a two-hemistich line of poetry:
\startitemize[packed]
\item {\bf line} forces typesetting of the two half-lines next to each
  other with the desired distance between them.
\item {\bf staggered} typesets the two half-lines on different lines
  with the first flushleft and the second flushright.
\item {\bf stacked} typesets the two half-lines on top of each other.
\item {\bf auto} picks between the first two modes preferring the first one.
\stopitemize

Another capability is the \type{width=fit} option.  Here we are not
restricted to a fixed width or \type{width=local}.  This or the
fixed-width option can be combined with \type{location=middle} to
place the whole box of poetry horizontally centered on the page.

Though this is my preferred settings, it comes with a caveat---a bug I
hope to fix soon: The line may not fit the local width, and then we'd
switch to \type{alternative=staggered}, which then reduces the maximum
line width to that of one hemistich, in effect leading to an
\type{alternative=stacked} mode.

In typesetting Persian poetry, we sometimes have alternative forms.
In the \emph{mostazad} poetry form, each line consists of two
half-lines, though of different width. This can of course be easily
typeset with any table mechanism, say Natural Tables. However, we add
the option of providing different \type{coupling} parameters for
half-lines of poetry. In the case of mostazad, we use something like
\type{coupling:right=extra}, leaving \type{coupling:left} to fall back
on \type{coupling=default}.

Other types of Persian poetry (for example, Tarkib-Band) consists of
differnet sections. Each section has a number of lines where all half
lines have the same width; then the section ends with a line of
different width, meter, and usually shape. For instance, the normal
lines are typically typeset as \type{alternative=line} but the
separating line is often typeset as \type{alternative=stacked}.
Whether the width of these separator lines (SL) are also match to the
normal lines (NL) may vary.
\startitemize[n]
\item Some would use the same \type{alternative=line} and match the
  widths.
\item Some would use the same \type{alternative=stacked} and match the
  widths.
\item Some would use the same \type{alternative=stacked} and not match
  the widths. Specially in this case, it's possible that widths of
  lines from different sections are matched or not. One may even want
  to match the width of NLs across sections and the width of the SLs
  together.
\stopitemize

There's yet another type of classic Persian poetry (called mosammat)
which can consist of odd number of lines. In this case, the last
half-line is usually middle-aligned.

To demonstrate thee options, we use Persian poetry.

In this example, the same line of poetry is typeset with the four
alternatives while using \type{width=fit} and \type{location=middle}.

%% Common definition for all the Persian poetry settings below.
\setuppoetry[before={%
    \setupalign[r2l]%
    \switchtobodyfont[dabeerformal]%
    \blackrule[width=\textwidth,height=.2ex]\page[no]},
  after={\page[no]\blackrule[width=\textwidth,height=.2ex]}]
%%

\startpoetry[color=red,%location=right,
  location=middle,
  %kashida=on,
]
  \poetryline[alternative=line]
    {میازار موری که دانه‌کش است}
    {که جان دارد و جان شیرین خوش است}

  \poetryline[alternative=staggered]
    {میازار موری که دانه‌کش است}
    {که جان دارد و جان شیرین خوش است}
  
  \poetryline[alternative=stacked]
    {میازار موری که دانه‌کش است}
    {که جان دارد و جان شیرین خوش است}

  \poetryline
    {میازار موری که دانه‌کش است}
    {که جان دارد و جان شیرین خوش است}
\stoppoetry

We demonstrate the same poetry line with the \type{kashida=on} option.
Note that kashida is inserted in random locations of the first
half-line (because the second half-line is wider and determines the
maximum width).

\startpoetry[color=blue,
  location=middle,
  kashida=on,
]
  \poetryline[alternative=line]
    {میازار موری که دانه‌کش است}
    {که جان دارد و جان شیرین خوش است}
 
  \poetryline[alternative=stacked]
    {میازار موری که دانه‌کش است}
    {که جان دارد و جان شیرین خوش است}

\stoppoetry

Next comes an excerpt from the famous TarkibBand by Mohtasham Kashani.
Here we let different sections match the width on the normal and separator lines.

\define\skippedlines{\poetryline[n=1,alternative=middle,coupling=misc]{\definedfont[dejavusans]⋮}}

\definepoetry[TarkibBand][distance=1cm,color=orange,kashida=on,width=local]
\definepoetryline[Band][coupling=band,alternative=stacked,color=red]
\startTarkibBand
\poetryline
{باز این چه شورش است که در خلق عالم است}
{باز این چه نوحه و چه عزا و چه ماتم است}

\poetryline
 {باز این چه رستخیز عظیم است کز زمین}
 {بی‌ نفخ صور خاسته تا عرش اعظم است}

\skippedlines

\poetryline
  {جن و ملک بر آدمیان نوحه می‌کنند}
  {گویا عزای اشرف اولاد آدم است}

\Band
{خورشید آسمان و زمین، نور مشرقین}
{پروردهٔ کنار رسول خدا حسین}

\poetryline
{کشتی‌شکست‌خوردهٔ طوفان کربلا}
{در خاک و خون تپیدهٔ میدان کربلا}

\poetryline
{گر چشم روزگار بر او فاش می‌گریست}
{خون می‌گذشت از سر ایوان کربلا}

\skippedlines

\poetryline
{آه از دمی که لشکر اعدا نکرده شرم}
{کردند رو به خیمهٔ سلطان کربلا}

\Band
{آن دم فلک بر آتش غیرت سپند شد}
{کز خوفِ خصم، در حرم، افغان بلند شد}

\stopTarkibBand

We now demonstrate that Kashida works with diacritics as well,
using the most well-known lines from Hafez.

\definepoetry[Ghazal][color=magenta,kashida=on,width=local]
\startbuffer[hafez]
  \poetryline
    {اَلا یا اَیُّهَا السّاقی اَدِرْ کَأسَاً و ناوِلْها}
    {که عشق آسان نمود اوّل ولی افتاد مشکل‌ها}
  \poetryline
    {به بویِ نافه‌ای کآخِر صبا زان طُرّه بُگشاید}
    {ز تابِ جَعدِ مشکینش چه خون افتاد در دل‌ها}
\stopbuffer
\startGhazal
  \getbuffer[hafez]
\stopGhazal


Here's a mosammat from Naser Khosrow with three lines per section.

\definepoetry[Mosammat][width=local,distance=5mm,color=darkyellow,kashida=on]
\definepoetryline[Tak][n=1]

\startMosammat
\poetryline
  {ای گنبد زنگارگون، ای پرجنون پرفنون}
  {هم تو شریف و هم تو دون، هم گمره و هم رهنمون}

\Tak
{دریای سبز سرنگون، پر گوهر بی منتهی}

\poetryline
{انوار و ظلمت را مکان، بر جای و دائم تازنان}
{ای مادر نامهربان، هم سالخورده هم جوان}

\Tak
{گویا ولیکن بی زبان، جویا ولیکن بی‌وفا}

\poetryline
{گه خاک چون دیبا کنی، گه شاخ پر جوزا کنی}
{گه خوی بد زیبا کنی، از بادیه دریا کنی}

\Tak
{گه سنگ چون مینا کنی، وز نار بستانی ضیا}
\stopMosammat

And finally a mostazad by MohammadTaghi Bahar, which barring the
kashida support, could have been easily typeset by any table
mechanism.

\definepoetryline[Mostazad][couplingright=extra,alternative=line]
\startpoetry[color=darkgreen,
  kashida=on,location=middle]
  \Mostazad
    {این دود سیه‌فام که از بامِ وطن خاست}
    {از ماست که بر ماست}
  \Mostazad
    {وین شعله سوزان که برآمد ز چپ و راست}
    {از ماست که ‌بر ماست}
  \Mostazad
    {جان گر به لب ما رسد، از غیر ننالیم}
    {با کس نسگالیم}
  \Mostazad
    {از خویش بنالیم که جانِ سخن اینجاست}
    {از ماست که ‌بر ماست}
\stoppoetry


\stoptext